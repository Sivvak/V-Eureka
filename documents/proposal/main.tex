\documentclass[10pt,twocolumn]{article}
\usepackage[utf8]{inputenc}
\usepackage{geometry}
\usepackage{graphicx}
\usepackage{amsmath,amssymb}
\usepackage{hyperref}

\geometry{
  top=0.65in,
  bottom=0.65in,
  left=1in,
  right=1in
}

\title{\textbf{Multimodal Eureka: Rewards Generated by LLM with Additional Visual Input}}
\author{
\begin{tabular}{cc}
     \textbf{Łukasz Łopacki} & \textbf{Szymon Pobłocki} \\
     Unversity of Warsaw & University of Warsaw \\
     ll439936@students.mimuw.edu.pl & sp438679@students.mimuw.edu.pl
     \\\\
     \textbf{Paweł Siwak} & \textbf{Jakub Winiarski} \\
     Unversity of Warsaw & University of Warsaw \\
     pl.siwak@student.uw.edu.pl & jj.winiarski@student.uw.edu.pl
\end{tabular}
\\\\
\textbf{Konrad Staniszewski\footnote{Project supervisor}} \\
University of Warsaw \\
ks.staniszewski@uw.edu.pl
}
\date{\today}

\begin{document}

\maketitle

\begin{center}
    {\Large\bfseries Abstract}
\end{center}

\begin{center}
\begin{minipage}{0.90\linewidth}
\small

...

\end{minipage}
\end{center}

\section{Introduction}
...

\section{Related Work}
...

\section{Experiments}
The main goal of our project is to build upon ideas presented in the \textit{Eureka} paper~\cite{eureka}. We would like to leverage the multimodal abilities of current SoTA LLMs and try to get even higher efficiency than in the referenced paper.

In the original work, \textit{Eureka} was compared with L2R~\cite{l2r}, human-written rewards created by active reinforcement learning researchers, and sparse rewards (success/failure). \textbf{The main purpose of our experiments is to compare \textit{Multimodal Eureka} with the original \textit{Eureka} itself.} Because \textit{Eureka} has already been compared with the aforementioned baselines, we do not see a reason to check other baselines. We aim to achieve better results than in the original paper by leveraging the power of computer vision, particularly in terms of spatial understanding.

To check the results, we will conduct a series of experiments.

\begin{itemize}
    \item We will calculate \textit{Success Rate} for both algorithms for multiple environments, including those used in the original paper (e.g. \textit{Isaac Gym}~\cite{isaac_gym}).
    \item We will test different types of prompts for image analysis in \textit{Multimodal Eureka} and choose the best one in terms of the average \textit{Success Rate}.
    \item The experiments will probably require a large model, so our choice of LLM will depend on what we can access. We will consider using multimodal \textit{Gemini} and \textit{GPT} models.
    \item We will experiment with different numbers of input images (e.g. last few frames).
    \item We will check the impact of image quality on the \textit{Success Rate}.
\end{itemize}

Our goal is to find the best possible solution. The primary challenge lies in creating prompts that will enable the model to extract the most crucial information from the images. The second challenge is to make the solution as general as possible. The original \textit{Eureka} uses the same set of prompts for each environment, and we would like to fulfill this requirement too.

To explain to the reader what the aim of \textit{image prompts} is, let us show an example of what a simplified version of a prompt could look like:
\\\\
\texttt{Based on an input image and instructions from the initial\_system prompt~\footnote{initial\_system prompt is a fragment from source code repository~\cite{eureka_repo} describing an agent his role and expected output}, create an optimal reward function, which will allow the agent to solve the environment efficiently. Take into consideration the laws of physics and try to connect variable names from the source code with what you see in the provided pictures.}
\\\\
Different prompts can lead to various unexpected behaviors. Therefore, we will try to identify such cases and list the most interesting observations. If we get meaningful results for situations with a clear physical explanation, we will list them in the final report as well.


\begin{thebibliography}{9}
\bibitem{eureka}
Yecheng Jason Ma and William Liang and Guanzhi Wang and De-An Huang and Osbert Bastani and Dinesh Jayaraman and Yuke Zhu and Linxi Fan and Anima Anandkumar, \emph{Eureka: Human-Level Reward Design via Coding Large Language Models}. arXiv preprint: Arxiv-2310.12931, 2023
\bibitem{l2r}
Wenhao Yu, Nimrod Gileadi, Chuyuan Fu, Sean Kirmani, Kuang-Huei Lee, Montse Gonzalez Arenas, Hao-Tien Lewis Chiang, Tom Erez, Leonard Hasenclever, Jan Humplik, et al. \emph{Language to rewards for robotic skill synthesis}. arXiv preprint arXiv:2306.08647, 2023.
\bibitem{isaac_gym}
Viktor Makoviychuk, Lukasz Wawrzyniak, Yunrong Guo, Michelle Lu, Kier Storey, Miles Macklin,
David Hoeller, Nikita Rudin, Arthur Allshire, Ankur Handa, et al. \emph{Isaac gym: High performance
gpu-based physics simulation for robot learning}. arXiv preprint arXiv:2108.10470, 2021.
\bibitem{eureka_repo}
Yecheng Jason Ma and William Liang, \emph{Eureka Repository on GitHub}, https://github.com/eureka-research/Eureka
\end{thebibliography}

\end{document}